% ATSR CD-ROM Booklet for Distribution Package
%
% Written September 1995 by Jo Murray, Space Science Department, RAL.
%
%\documentclass[10pt,]{article} 
%\usepackage{times}

\documentstyle [twoside] {article}
%
\pagestyle{myheadings}
\markboth{Rutherford Appleton Laboratory}
{ATSR CD-ROM}

\setlength{\textwidth}{4.0in}
\setlength{\textheight}{3.8in}
\setlength{\topmargin}{0.2in}
\setlength{\oddsidemargin}{0.5in}
\setlength{\evensidemargin}{0.5in}
\setlength{\parindent}{0pt}
\setlength{\parskip}{\smallskipamount}
\raggedbottom
\tolerance=800
%
% Quire input and settings (see quire.doc for more information)
\input quire
\latexquire
%\htotal 5.0in       \vtotal 5.0in
\htotal 4.75in       \vtotal 4.7in
\horigin -0.3in     \vorigin -0.2in
\shoutline 0pt      \shstaplewidth 0pt
\shcrop 1pt
%\shonly2
% Use these two lines to format one booklet page per sheet with crop marks
\shhtotal 8.5in
\qonepage
% Use these two lines to format two booklet pages per sheet with crop marks
% (Make sure the printer is set for landscape mode, not portrait mode)
%\shhtotal 11in
%\quire{16}
%
\newcommand{{\mic}}{{$\mu$m}}
\newcommand{{\etal}}{{\sl et al.}}
\newcommand{{\degree}}{{$^{\rm o}$}}
\newcommand{\dir}[1]{$[${\bf#1}$]$}
\newcommand{\tm} {$^{\rm\small TM}$} 
\hfuzz=4pt

\title{ \bf {\Huge Sea-surface temperatures from ATSR \\}
{\sl \LARGE \sl August 1991 -- July 1995}}
\author{ Jo Murray \thanks{Telephone (01235)~446705; Fax (01235)~445848;
e-mail  jo.murray@rl.ac.uk.} \\ \\ 
Space Science Department \\ Rutherford Appleton Laboratory 
\date{September 1995}
}

\begin{document}
\maketitle
\clearpage

\tableofcontents
\listoffigures
\listoftables


\clearpage


 

\section{Introduction}
\label{sec:intro}

The first Along-Track Scanning Radiometer (ATSR-1) is a four-channel, 
dual-view, self-calibrating, infrared radiometer with spatially coregistered 
spectral channels centred at  1.6{\mic}, 3.7{\mic}, 10.8{\mic} and 12.0{\mic}.
The instrument's innovative design has enabled the remote sensing of 
Sea-Surface Temperature (SST) to unprecedented levels of accuracy 
(better than $0.3$K).
The instrument was launched on board ESA's first remote-sensing 
satellite, ERS-1, on 21st  July 1991, and has performed well since that 
time with the only serious setback being the failure of the 3.7\mic\  
channel in May 1992.

The instrument design is described in Edwards \etal\ (1990)
and the results from the validation programme  are reported in 
Mutlow \etal\ (1995) and  Barton \etal\ (1995).
A short description of the novel aspects of ATSR is given 
in Appendix~\ref{app:atsr}.

Suffice to say here that the on-board calibration and the use of the 
along-track scanning technique enable the accurate retrieval of ocean skin
temperature.\footnote{Infrared radiometers measure the radiant 
temperature of a water layer 
less than 0.1{\thinspace}mm thick; this is typically 0.5K cooler than 
the bulk temperature measured at 1{\thinspace}m depth.}
Unlike the AVHRR SST retrieval algorithms which are
empirically adjusted using {\sl in situ\/} bulk measurements 
(McClain \etal, 1985) 
the ATSR scheme results in the determination of very accurate SSTs without 
recourse to ground-based observations.
Therefore the data presented here represent an independent set of 
measurements of global SST.

The second instrument in the ATSR series, ATSR-2, was launched on board 
ESA's second remote-sensing satellite, ERS-2, on 21st April 1995.
ERS-2 is currently (September~1995) in a similar orbit but trailing  one 
day behind ERS-1.
ATSR-1 has already operated beyond its three-year design lifetime  
but is likely to remain in operation at least until the end of 1995.
This will enable comparison of data from the two ATSRs, and 
data from both instruments will be made available on a future CD-ROM set.

Four years of data,  August 1991 -- July 1995 inclusive, are 
supplied on this CD-ROM set.
Updates to the data, documentation and software presented here will 
be made available on the anonymous ftp account {\tt sstaxp.ag.rl.ac.uk}
(username {\tt anonymous}, and please supply your e-mail address as 
the password).
The {\tt atsr1} directory  on this account has a subdirectory
structure which reflects the CD-ROM layout as described in 
Section~\ref{subsec:layout}.


\subsection{ATSR data products}
\label{subsec:introdata}

A variety of products are derived from ATSR data, but this CD-ROM set 
comprises only two types of data.
\begin{itemize}
\item The spatially-Averaged Sea-Surface Temperature (ASST) product.
These data comprise half-degree, spatially-averaged SSTs 
with associated temporal, positional  and confidence 
information as described in Section~\ref{sec:asst}.

\item Time-averaged global maps.
\begin{enumerate} 
\item A monthly average at half-degree spatial resolution as 
described in Section~\ref{subsec:month}.
\item A five-day average at one-degree spatial resolution as 
described in Section~\ref{subsec:super}.
\end {enumerate}
\end {itemize}


Other data products from ATSR include: high-resolution 
({\sl i.e.\ }1km$\times$1km\/) `brightness temperature' products 
for each thermal channel; 
%high-resolution `reflectance' products from the 1.6\mic\ channel;
high-resolution SSTs, and  products containing results 
from cloud identification tests performed during processing.
A complete description of these and other ATSR data products 
is given in Bailey (1993).

\subsection{CD-ROM layout}
\label{subsec:layout}


This CD-ROM set is written to the ISO~9660 standard with no record format
specified for the files.
Each end of line in the text files is marked by a line-feed character.
On VMS systems where the default action is to use carriage-return 
to indicate ends of lines,  the line structure of text files will look strange 
and the text will contain embedded control characters. 
However the files will remain readable.
Advice for dealing with this and other complications on VMS systems is 
given in the file {\tt vms.txt} in the \dir{docs} directory.

The structure of the CD-ROM set is summarised in Figure~\ref{fig:layout}.
In addition to those containing data, there are directories for software
and documentation.

\begin{description}
\item{\bf Software:} -- the \dir{software} directory 
contains programs to assist in reading the supplied datasets
as described in Section~\ref{sec:soft}.
All data are stored as 2-byte or 4-byte integers
in fixed-length records for maximum portability and ease of use.
Details of the actual formats are given in Sections~\ref{sec:asst}
and~\ref{sec:time} for the ASST data and time-averaged global maps respectively.
\item{\bf Documentation:} --
as described in Section~\ref{sec:docs},
the \dir{docs} directory contains the \LaTeX\ source 
for this booklet and various other useful text files.
\end{description}

\clearpage
\begin{figure}\caption{Arrangement of data on CD-ROM set}
\label{fig:layout}
\bigskip
\smallskip

\begin{picture}(200,250)

\put(5,195){\framebox(40,15){\bf DISK 1}}

\put(45,200){\line(1,0){5}}

\put(50,200){\vector(1,0){30}}

\put(50,200){\line(0,-1){30}}
\put(70,170){\oval(40,40)[bl]}
\put(70,150){\vector(1,0){5}}

\put(50,200){\line(0,1){30}}
\put(70,230){\oval(40,40)[tl]}
\put(70,250){\vector(1,0){10}}


\put(85,245){\framebox(25,10){\tt asst}}
\put(85,195){\framebox(25,10){\tt docs}}
\put(75,145){\framebox(50,10){\tt software}}

\put(140,247){\sl ASST data for 1991--1993}
\put(140,197){\sl documentation}
\put(140,147){\sl software}



%************************ -100

\put(5,45){\framebox(40,15){\bf DISK 2}}

\put(45,50){\line(1,0){5}}

\put(50,50){\vector(1,0){23}}

\put(50,50){\line(0,-1){30}}
\put(70,20){\oval(40,40)[bl]}
\put(70,00){\vector(1,0){5}}

\put(50,50){\line(0,1){30}}
\put(70,80){\oval(40,40)[tl]}
\put(70,100){\vector(1,0){10}}


\put(85,95){\framebox(25,10){\tt asst}}
\put(80,45){\framebox(37,10){\tt months}}
\put(75, -5){\framebox(50,10){\tt supers}}

\put(140,98){\sl ASST data for 1994--1995}
\put(140,48){\sl Monthly Maps}
\put(140, -2){\sl Five-day maps}
\end{picture}
\end{figure} 

\clearpage


\section{The spatially-Averaged Sea-Surface Temperature (ASST) product}
\label{sec:asst}

The ASST product contains half-degree, spatially-averaged SSTs 
with associated temporal, positional  and confidence 
information.
These data are generated using SADIST 
(Synthesis of ATSR Data Into Sea-surface Temperatures),  the 
Rutherford Appleton Laboratory's ATSR data-processing scheme
(Z\'{a}vody \etal, 1994, 1995).
This software is under continuous development and the data described herein 
were produced by SADIST, Version~500, hereafter SADIST~v500 
(Bailey, 1993).
See Appendix~\ref{app:sad2} for a discussion of planned changes to
SADIST and the ASST product.

\subsection{ASST derivation} 
\label{subsec:sst_types}

As described in Appendix~\ref{app:sad2}, SST retrieval involves
a linear combination of single-channel brightness temperatures
from both forward and nadir views.
Derivation of half-degree ASSTs is accomplished in two steps:
\begin{enumerate}

\item Single-pixel (1km$\times$1km) brightness temperatures are 
averaged into ten-arcminute brightness temperatures.
SSTs are then derived from these brightness temperatures via the usual 
retrieval algorithm.

\item These ten-arcminute SSTs are then averaged to obtain a half-degree SST. 
Nine ten-arcminute cells are contained in each half-degree (thirty-arcminute)
cell.
\end{enumerate}

A proportion of the ten-arcminute, nadir-view brightness temperatures is 
without associated forward-view values.
This is usually  a result of cloud coverage affecting only the forward
scene, but may occur because for sections of most orbits, the forward and
nadir scenes are stored on different raw data tapes.
Including such nadir-only data compromises the SST accuracy,
whereas omitting these data results in reduced coverage.

Rather than attempt a single solution to this problem, 
three different SSTs are provided for each half-degree cell 
as described below.
\begin{description}
\item[\bf Nadir-view SSTs:] SSTs derived using the nadir-view data only.
\item[\bf Dual-view SSTs:] SSTs derived using both nadir- and forward-view
data.
\item[\bf Mixed SSTs:] SSTs derived using a dual-view retrieval if possible,
and nadir-only retrieval otherwise.
This will coincide with the nadir-view SST when no forward data are available 
for a half-degree cell, and with the dual-view SST  in the case where 
all the contributing ten-arcminute cells have associated forward data.
\end{description}

The choice of SST type to use will vary with application.
Dual-view SSTs represent the most accurate retrieval, and are least 
likely to be affected by cloud contamination.
However exclusive use of these data will restrict coverage.
For example in 1992, a dual-view SST was present for only 85\% of those
half-degree cells for which an SST had been evaluated.

\subsection{Arrangement of ASST data on CD-ROMs}
\label{subsec:arr}

The ASST data are stored in the directory \dir{asst} on the CD-ROM
set. Disk~1 contains data from 1st~August 1991 to 31st December 1993
inclusive.
Data for the period 1st~January 1994 to 31st~July 1995 are contained in 
the \dir{asst}  directory on Disk~2.

Each day's data is in a separate file; the filenames have the form:\\
\hspace*{2cm} {\tt day }{\sl y mm dd}{\tt.dat}\\
where {\sl y} is the (year$-1990$), {\sl mm} is the month of the year,
and {\sl dd} is the day of the month.
Thus {\tt day20304.dat} contains data for 4th~March 1992.

Each file contains a number of 32-byte records, each of which 
is associated with a single half-degree cell.
The number of records per day is highly variable, and lies in the range
$0-31624$.
Days on which no reliable data were obtained (usually due to instrument operations)
are associated with files of zero length.
The variation in daily data volume during normal operation
reflects differences in the proportion of a day's coverage that was over 
cloud-free water.



\subsection{ASST product format}

 
The contents of the 32-byte record product are shown in Table~\ref{tab:record}
and described in full below.
Each record contains SSTs and associated information for a single 
half-degree cell.
The information is stored as two- or four-byte integers, but 
as an example, a textual representation of the first five records 
for 1st~January 1992 is shown below:

\begin{footnotesize}{\begin{verbatim}
15340  202  59 333  4   27130  -1  27143  -1   27136  -1  11   9351
15340  203  59 334  4   27123   4  27153   6   27153   6  29  43655
15340  221  60 337  2   27118  13  27165   6   27154  21  48  26759
15340  236  62 340  0   27141   7  27158  -1   27149   8  23   9863
15340  237  62 334  3   27127  -1  27161  -1   27161  -1  34   8839
\end{verbatim}}\end{footnotesize}
The first two numbers give the time (in days and seconds as described below), 
the third and fourth give the position of the half-degree cell 
(again, see below),
and the fifth number is associated with the position of the cell within
the ATSR swath.

The sixth, eighth and tenth values are ASST values in centiKelvin,
and the seventh, ninth and eleventh represent the associated 
root-mean-square (rms) deviations of the contributing ten-arcminute SSTs in each
case.

The last two values represent the mean view-difference and the confidence 
word as described below.
Any parameter that cannot be evaluated is represented as $-1$ 
(see Table~\ref{tab:exc}). 
A description of each parameter is given below, and the format summarised
in Table~\ref{tab:record}.

\begin{table}
\begin{center}
\caption{\bf ASST product record format }
\label{tab:record}
\begin{tabular}{|r|l|l|l|} \hline

{\bf Byte } & {\bf Parameter description} & {\bf Type} & {\bf Unit} \\ 
{\bf range} & & & \\ \hline \hline

0 -- 3   & Time of data (days since Jan 1st 1950)   & Integer & Days  \\ \hline
4 -- 7   & Time of data (secs within current day)   & Integer & Secs  \\ \hline
8 -- 9   & Latitude cell number                     & Integer & Cell  \\ \hline
10 -- 11 & Longitude cell number                    & Integer & Cell  \\ \hline
12 -- 13 & Mean across-track band number            & Integer & None  \\ \hline
14 -- 15 & Nadir-only ASST                          & Integer & K/100 \\ \hline
16 -- 17 & RMS deviation of ten-arcmin SSTs         & Integer & K/100 \\ 
         & contributing to nadir-only ASST          &         &       \\ \hline
18 -- 19 & Dual-only ASST                           & Integer & K/100 \\ \hline
20 -- 21 & RMS deviation of ten-arcmin SSTs         & Integer & K/100 \\
         & contributing to dual-only ASST           &         &       \\ \hline
22 -- 23 & Mixed nadir/dual ASST                    & Integer & K/100 \\ \hline
24 -- 25 & RMS deviation of ten-arcmin SSTs         & Integer & K/100 \\ 
         & contributing to mixed nadir/dual ASST    &         &       \\ \hline
26 -- 27 & Mean view-difference of nadir/dual       & Integer & K/100 \\
         & ten-arcminute SST                        &         &       \\ \hline
28 -- 31 & Confidence word associated with ASST     & None    & None  \\ 
         & derivation                               &         &       \\ \hline
\end{tabular}
\end{center}
\end{table}

\begin{description}

\item[Time of data (days).] This is given as the 
number of days since 1st~January, 1950 (Modified Julian Day) and 
does not include the current, incomplete day.
Dates in the ({\sl y mm dd}) format (as described in 
Section~\ref{subsec:arr}) are given for the relevant day numbers 
in the text file {\tt daydates.dat} in the \dir{docs} directory on Disk~1 
(see Section~\ref{sec:docs}).

\item[Time of data (seconds).] The integer value containing the time in 
seconds within the current day.
Note that the time used within each record is the time of the first 
ATSR nadir-view
scan to contribute to the ASST derivation. 
The variable nature of cloud-cover makes it
impossible to predict the position of this scan relative to the centre of
the half-degree cell. Under any circumstances, this time cannot be more
than approximately six seconds from the time at which the centre of the
cell is scanned by the nadir view. 

\item[Latitude.] The latitude is provided as a cell number. The edges of
half-degree cells are sections of parallels and meridians. The latitude
cells are numbered from the South Pole to the North Pole, in the range 0 to
359. Latitude cell number 0 extends from 90{\degree}S to 89.5{\degree}S; 
latitude cell number 359 extends from 89.5{\degree}N to 
90{\degree}N. The latitude of the cell centre may be derived by:
\begin{quote} ${\sl latitude} = 
(({\sl lat}\_{\sl cell}\_{\sl num} - 180.0) / 2.0) + 0.25$.
\end{quote} 
The latitudes which are implied by the product cell numbers are
geocentric. 
Geodetic latitudes may be derived using:
\begin{quote}
$ {\sl geodetic} = tan^{-1}(1.0067451 \times tan({\sl geocentric}))$
\end{quote}

\item[Longitude.] The longitude is provided as a cell number. The edges of
half-degree cells are sections of parallels and meridians. The longitude
cells are numbered from 180{\degree}W to 180{\degree}E, in the range
0 to 719. Longitude cell number 0 extends from 180{\degree}W to 
179.5{\degree}W; longitude cell number 719 extends from 179.5{\degree}E to
180{\degree}E. The longitude of the cell centre may be derived by:
\begin{quote} $longitude = ((lon\_cell\_num - 360.0) / 2.0) + 0.25$.
\end{quote} 

\item[Mean across-track band number.] The five across-track bands are numbered
0 to 4 from left to right,  and are centred on the ground-track. 
Each band is 50{\thinspace}km 
wide, except the fifth, which is 62{\thinspace}km wide, and extends to the 
edge of the swath. 

                   
\item[Nadir-only SST and rms deviation.] These are the mean and the rms
deviation of ten-arcminute cell SSTs derived using only
nadir-view data. Note that the rms deviation is set to $-1$ if fewer than three
ten-arcminute cells contributed. 

\item[Dual-only SST and rms deviation.] These are the mean and the rms 
deviation of ten-arcminute cell SSTs derived using
nadir- and forward-view data. Note that the SST is set to $-1$
if dual-view retrieval was not possible for any ten-arcminute cell, and
that the rms deviation is set to $-1$ if fewer than three ten-arcminute cells
contributed. 

\item[Mixed SST and rms deviation.] These are the mean and the rms  deviation
of ten-arcminute cell SSTs derived using the best
available combination of nadir- and forward-view data. Note that the
rms deviation is set to  $-1$ if fewer than three ten-arcminute cells 
contributed.
If no ten-arcminute cells were used by the dual-view retrieval, these
values will be identical to the nadir-only SST and rms deviation.

\item[Mean view-difference.] This is the mean difference between the
ten-arcminute cell SSTs derived using dual-view and
nadir-only retrieval algorithms. This value is set to $-1$ if no dual-view 
retrieval was possible. 

\item[Confidence word.] Table~\ref{tab:con} describes the contents of the
ASST product confidence word. 
Bits 0 to 8 are set if, within all of the contributing ten-arcminute cells,
at least 90\% of the contributing pixels have the
characteristic defined in Table~\ref{tab:con}. 
Bits 9 to 12 contain the number of ten-arcminute cell SSTs 
contributing to the nadir-only half-degree SST, 
and (the number is necessarily the same) to the mixed
nadir-only/dual-view half-degree SST.
Bits 13 to 16 contain the number of ten-arcminute cell SSTs 
contributing to the dual-only half-degree SST. 
This number may be zero. 


\end{description}

\begin{table}
\caption{\bf ASST product confidence word}
\label{tab:con}
\begin{center}
\begin{tabular}{|r|l|} \hline

{\bf Bits } & {\bf Meaning if set}                          \\ \hline \hline
0           & 12.0$\mu$m channel present in source data     \\ \hline
1           & 11.0$\mu$m channel present in source data     \\ \hline
2           & 3.7$\mu$m channel present in source data      \\ \hline
3           & 1.6$\mu$m channel present in source data      \\ \hline
4           & Cloud-identification used 1.6$\mu$m histogram \\
            & reflection cloud test                         \\ \hline
5           & 1.6$\mu$m histogram reflectance cloud test    \\
            & used a dynamic threshold                      \\ \hline
6           & Sunglint detected by 1.6$\mu$m                \\
            & reflectance reflectance cloud test            \\ \hline
7           & 3.7$\mu$m channel used in SST retrieval       \\ \hline
8           & SST derivation used day-time data             \\
            & (night-time if zero)                          \\ \hline
9 -- 12     & Number of ten-arcminute cell SSTs             \\
            & derived using nadir-only retrieval (1 -- 9)   \\ \hline
13 -- 16    & Number of ten-arcminute cell SSTs             \\
            & derived using dual-only retrieval (0 -- 9)    \\ \hline
17 -- 31    & Unused                                        \\ \hline
\end{tabular}
\end{center}
\end{table}


Exceptional values within the ASST product are described in 
Table~\ref{tab:exc}. 

\begin{table}
\begin{center}
\caption{\bf ASST product exceptional values}
\label{tab:exc}
\begin{tabular}{|r|l|l|} \hline

{\bf Parameter}      & {\bf Value} & {\bf Reason}              \\ \hline \hline

Dual-only SST        & $-1$        & No ten-arcminute cells used a \\
                     &             & dual-view SST retrieval       \\ \hline
RMS deviations       & $-1$        & Fewer than 3 ten-arcminute    \\
                     &             & cells contributed to the      \\
                     &             & half-degree ASST              \\ \hline
Mean view-difference & $-1$        & No ten-arcminute cells used a \\
                     &             & dual-view SST retrieval       \\ \hline
\end{tabular}
\end{center}
\end{table}



\clearpage
\subsection{Quality control of ASST data}
\label{subsec:qa}

The quality of SSTs from ATSR data can be compromised by a number of causes.
These can be classified broadly into two groups as follows:
\begin{enumerate}
\item ATSR or ERS-1 instrument operations.
\item Inadequate cloud-clearing or atmospheric correction.
\end{enumerate}

Clearly data which are unreliable as a result of the first cause should not
be included in this archive, and the quality-control procedure described
below is addressed to removing these data.


However some data affected by cloud-contamination or high aerosol levels remain
in the archive. 
Removal of these data would be inappropriate as the selection process 
would require arbitrary choices and
would exclude the use of the archive for many purposes, such as 
studying the  effect of aerosols or the effectiveness of the ATSR 
cloud-clearing scheme\footnote{Jones \etal\ (1995) have recently 
suggested a useful scheme for identifying cloud-contaminated ASST data.}.
See Appendix~\ref{app:sad2} for more information on the ATSR 
cloud-identification scheme.

Data arising from periods when ATSR-1 was not in nominal operation have been 
identified in two ways:

{\bf Inspection of the ATSR operations log:} -- 
various instrument operations affect data quality.
For example, periodically the instrument cooler is turned off to allow
residues which have accumulated on the detectors to boil off.
During this `outgassing' procedure, the thermal detectors lose their
sensitivity and the associated data are of little value.
Orbit manoeuvres also affect data quality with the forward-view data
being particularly affected.
The ATSR operations log is 
stored in {\bf atsrops.txt} in the \dir{docs} directory.
Note that not all suspicious data have been discarded. For example
data from the commissioning phase (from launch until 14th September 1991) 
are regarded as non-nominal, but most of these data remain in the archive.


{\bf Inspection of the deviation from climatology:} --
less serious instrument operations can also affect the data, and these 
cannot always be identified from the instrument 
log\footnote{In versions of the processing software previous to SADIST~v500, 
an occasional orbit was mis-geolocated due to an incorrect time being 
associated with the data; this problem has been corrected in the 
SADIST~v500 software.}.
Therefore it is necessary to compare data with climatological values
in order to identify such compromised observations.

GOSTA monthly one-degree climatology was used (see Bottomley \etal, 1990) with 
unfilled locations being assigned a value by taking the first of these to
provide a value:
\begin{enumerate}
\item the average of 8 adjacent points;
\item the average of 16 next-most adjacent points;
\item the zonal mean for that latitude;
\item 271.5K (only Arctic and  Antarctic regions have no zonal mean).
\end{enumerate}
Linear interpolation was used to generate a half-degree grid for 
comparison with the ASST data.
                    
Data have been discarded from periods  where:
\begin{itemize}
\item more than 10\% of the day-time ASSTs fell outside climatological 
values.
\item more than 40\% of the night-time ASSTs fell outside climatological 
values.
\end{itemize}

These thresholds were chosen after inspection of the percentage of
data per 100-minute orbit which was more than 6K from climatological
values.
The higher threshold selected for night-time observations arises
as these are more likely to be depressed as a result of incomplete 
cloud-clearing (in the absence of the 1.6\mic\ data at night).

The identification of a `bad' orbit did not necessarily result
in the discarding of 100-minutes worth of data.
Rather, for each such orbit,  the deviation of individual ASSTs 
from climatology was examined to establish when normal
operation ceased and resumed.
All data from periods which failed the aforementioned quality criteria 
have been removed from the archive.
All other data remain, including individual ASSTs widely deviant from 
climatology.


\begin{figure}
\caption{ATSR comparison with climatology in 1993}
\label{fig:clim}
\vspace{10cm}
\end{figure}

In general this test discriminated well and less than 0.3\% of ASST data has
been discarded.
Figure~\ref{fig:clim} shows the results from comparing day-time 
data with climatology for each month in 1993. The percentage of data in 
a 100-minute period (corresponding to one orbit) is shown as a function of time.
Note there is little sign of bad data until late June, when a 
bad period is clearly identifiable. This corresponds to an outgassing 
which was scheduled for 29th~June to 1st~July, and the affected data 
have been removed.
However the data also exhibit a marked increase in deviation from
climatology throughout July;
this is due to the effects of Saharan dust during that month, and, 
of course, such data remain in the archive.

The time ranges which have been removed from the supplied data set
are given in {\tt baddata.txt} in the \dir{docs} directory.


\subsection{Removal of duplicates from ASST archive}

A duplicate is defined to occur when more than one day-time observation or 
more than one night-time observation is recorded for a particular 
half-degree cell on a particular day.
Such duplicates arise in several ways as described below.
\begin{enumerate}
\item The same data have reached the RAL ASST archive 
more than once, for example,  after having been received from different 
ground stations. 
In this case, the ASST records are identical.

\item  More than one observation actually occurred. This is possible at high
latitudes.

\item Data division -- ATSR raw data are downlinked to a ground station 
about once per orbit, giving rise to a break in the data stream as presented
to the processing software.
No attempt is made to unite divided data for processing.
Therefore data which are associated with a single half-degree cell can 
be separated, and give rise to two distinct ASSTs, constructed with
complementary fractions of the complete data associated with the cell.
Such duplicates tend to occur at characteristic latitudes associated with
data downlink and usually several adjacent cells at that latitude are affected.

\item  Due to a minor bug in the processing software, occasionally 
two records are generated for a single half-degree cell. In such cases,
more than 90\% of the data is used for the first record and less than 10\% for
the second.

\end{enumerate}


In all these cases, the observation associated with the earlier time 
is retained and any subsequent record is removed.
Thus a maximum of one day-time and one night-time observation is 
stored for each half-degree cell for any day.
Clearly this is appropriate in the first case above where the same
data are represented more than once.
However in the second case (two distinct observations) both 
records are equally valid.
In the third and fourth cases, the later data are useful but
do not have equal weight with the first observation.

The duplicates arising from the latter three causes are 
stored in files in the \dir{asst} directories.
These are in the same format as the ASST files, with all the duplicates
for 1991--1993 in {\tt dups9193.dat} on Disk~1, and those for 1994--1995 in
{\tt dups9495.dat} on Disk~2, that is, the same temporal division used 
for the ASST data.


\clearpage

\section{Time-averaged global maps}
\label{sec:time}


Global maps with averaged SSTs are provided at two spatial 
and temporal resolutions as described below.

\begin{enumerate}
\item Monthly maps at half-degree resolution. 

\item Maps averaged  over five-day periods at one-degree resolution.
This is chosen to match the `superobservation' grid of the 
U.K.~Meteorological Office (UKMO), see Bottomley \etal, 1990.

\end{enumerate}

`Mixed' ASSTs  as described in Section~\ref{subsec:sst_types} were used.
Each averaged SST value is a simple unweighted mean of the day-time 
and night-time data associated with the appropriate half-degree or 
one-degree cell during the time period.
However only ASSTs which fell within 6K of climatology
(from Bottomley \etal, 1990)
contributed  to the averaged SSTs supplied in  these files.


Note that ATSR-1 does not achieve complete global coverage when ERS-1
is in a three-day repeat cycle. The repeat cycle history and the associated
global coverage are discussed in Appendix~\ref{app:orbit}.

\subsection{Monthly-averaged maps}
\label{subsec:month}

Data averaged over calendar months are 
stored on Disk~2 of the CD-ROM set in the directory
\dir{months}.

The filenames have the form:\\
\hspace*{2cm} {\tt month }{\sl y mm}{\tt.dat}\\
where {\sl y} is the (year$-1990$) and {\sl mm} is the month of the year.
Thus {\tt month203.dat} contains data for March 1992.

Each file contains a two-byte-integer array of dimensions $720\times360$.
Each array element represents an averaged  SST  in centiKelvin for a 
half-degree cell.
The terrestrial position associated with each SST is implicit in its 
array position as described below.
(The array is stored as 360 1440-byte records, with the 720 two-byte values
for a particular latitude stored in each 1440-byte record. 
Thus the first record contains values for the  half-degree cells
centred at 89.75{\degree}S {\sl etc}.)


Considering the array to be numbered $0\le i \le 719$, $0 \le j \le 359$,
the position of the cell centre of {\tt element}\dir{i,j} is given by:
\begin{quote} $longitude = ((i - 360.0) / 2.0) + 0.25$
\end{quote} 
\begin{quote} $latitude = ((j - 180.0) / 2.0) + 0.25$
\end{quote} 
The latitudes which are implied by the product cell numbers are
geocentric.

The exceptional value of $-1$ is used to indicate that no data were 
available to generate an ASST for a particular half-degree cell.
Such a value will apply to all cells over land, over ice, under cloud
when observed, or not observed at all during the month.


\subsection{Five-day averaged maps}
\label{subsec:super}

Data averaged over five-day periods at one-degree resolution are
stored on Disk~2 of the CD-ROM set in the directory
\dir{supers}.
This temporal and spatial resolution was chosen to match the UKMO's 
`superobservation' grid  (Bottomley \etal, 1990).
In leap years, data for 29th February are included with the previous 
five days, thus the superobservations start on the same dates from year to 
year.

The filenames have the form:\\
\hspace*{2cm} {\tt sup }{\sl y mm dd}{\tt.dat}\\
where {\sl y} is the (year$-1990$), {\sl mm} is the month of the year,
and {\sl dd} is the day of the month corresponding to the first day of the
five-day period.
Thus {\tt sup20307.dat} contains data for the five-day period 
starting on 7th~March 1992.


Each file contains a two-byte-integer array of dimensions $360\times180$.
Each array element represents an averaged  SST  in centiKelvin for a 
one-degree cell.
The terrestrial position associated with each SST is implicit in its 
array position as described below.
(The array is stored as 180 720-byte records, with the 360 two-byte values
for a particular latitude stored in each 720-byte record. 
The first record contains values for the  one-degree cells
centred at 89.5{\degree}S {\sl etc}.)

Considering the array to be numbered $0\le i \le 359$, $0 \le j \le 179$,
the position of the cell centre of {\tt element}\dir{i,j} is given by:
\begin{quote} $longitude = ((i - 180.0) / 2.0) + 0.5$
\end{quote} 
\begin{quote} $latitude = ((j - 90.0) / 2.0) + 0.5$
\end{quote} 
The latitudes which are implied by the product cell numbers are
geocentric.

The exceptional value of $-1$ is used to indicate that no data were 
available to generate an ASST for a particular one-degree cell.
Such a value will apply to all cells over land, over ice, under cloud
when observed, or not observed at all during the five-day period.


\clearpage


\section{Software }
\label{sec:soft}

Software to help with accessing the data on this  CD-ROM set 
is stored on Disk~1 in the directory \dir{software}.
Routines written in Fortran, C and 
IDL\footnote{IDL is a registered trademark of Research Systems, Inc.} are
provided as described below.

\begin{description}
\item{\tt readasst.c}  -- a C program to read an ASST file 
and print the contents of the first record. 

\item {\tt disasst.pro, dismonth.pro, dissuper.pro}  -- IDL procedures
to display the contents of ASST, monthly average and `superobservation' files 
respectively as global maps.

\item{\tt avgasst.for}  -- a FORTRAN program to produce a time-averaged
half-degree SST array for any date range.

\item{\tt rdmonth.for, rdsuper.for}   -- FORTRAN programs to read the
monthly and superobservation files respectively, and print some simple 
statistics about the data.
\end{description}

These programs are provided as templates from which the user can develop
more specialised software as required.
Whilst portability has been the main programming consideration,
inevitably small alterations to the software will be necessary.
In particular, the filenames of the input data must be 
changed to reflect the actual location of the CD directory 
on the local system.

The other likely change is associated with the byte-ordering used
by the computer system.
As mentioned earlier, the data are stored as 2-byte or 4-byte integers
in fixed-length records.
Byte ordering within values follows the 
DEC/VMS\footnote{VMS is a registered trademark of Digital Equipment Corp.}
standard; that is, bytes are ordered in increasing significance  
({\sl little-endian\/}).
This byte-ordering is also used on most PCs.
However, swapping of bytes will be necessary on those platforms 
where the contrary convention is used ({\sl big-endian}).
All the programs supplied incorporate a byteswap flag, and will perform
byte-swapping if this flag set to true.
Note that the programs will warn the user if the data values suggest the wrong 
choice regarding byteswapping has been made.

A complete description of the programs is  provided in the source code 
and in {\tt aareadme.txt} in the \dir{software} directory.

\clearpage
\section{Documentation}
\label{sec:docs}
Useful documentation stored in the \dir{docs} directory includes:
\begin{description}
\item {booklet.tex, booklet.ps} -- the \LaTeX\ source and
PostScript\footnote{PostScript is a registered trademark of ADOBE.}
file for this booklet.

\item {sad500.ps, sad2.ps} -- the PostScript files for the 
SADIST~v500 and SADIST~2 products documents.

\item {atsrops.txt} -- a text file containing the ATSR operations log.


\item{baddata.txt}  --  a text file summarising the time ranges which
have been removed from the ASST data set for quality control purposes
(see Section~\ref{subsec:qa}).
                     
\item{ftp.txt}  --  a text file describing the anonymous ftp
account where updates to the data, software and documentation will be 
made available.

\item{vms.txt}  --  a text file with VMS-specific advice.

\item {daydates.dat} -- dates in the ({\sl y mm dd}) format (as described in 
Section~\ref{subsec:arr}) are given for the day numbers as used in the ASST 
data.
\end{description}


\clearpage
\appendix


\section{The ATSR Instrument}
\label{app:atsr}

ATSR-1 is a dual-view, self-calibrating, infrared radiometer
with spectral channels centred at  1.6{\mic}, 
3.7{\mic}, 10.8{\mic} and 12.0{\mic} (Edwards \etal, 1990).
Several aspects of the ATSR design enable the measurement of 
high-precision SSTs.
\begin{description}
\item[\bf On-board calibration] -- two highly-precise, ultra-stable, 
on-board calibration
targets are  maintained at around 265K and 305K.
Radiation from each of these is measured during every scan,
and these data are used to continually calibrate the instrument, enabling
the determination of single-channel equivalent temperatures
correct to within 0.05K.
\item[\bf Low-noise detectors] -- a Stirling-cycle cooler maintains the
detectors below 100K.
At the start of the mission, noise temperatures lower than 0.05K 
(for scene temperatures up to 300K) were achieved for each 
channel.
\item[\bf Dual-view] -- ATSR's conical scanning technique results in two 
views of the Earth's surface, one close to nadir and one at 
55{\degree} to nadir.
As these views are associated with different atmospheric path
lengths, a correction for atmospheric absorption can be determined.
\end{description}
\clearpage

\section{SADIST -- the ATSR data processing scheme}
\label{app:sad2}

The tasks performed by the ATSR data-processing scheme include:
\begin{description}
\item {\bf Geolocation and Collocation of data: -- } 
the Earth location of acquired ATSR data is determined with an accuracy 
of better than 2{\thinspace}km, and the collocation (matching of forward 
and nadir data) has a similar accuracy (Z\'{a}vody, 1995).

\item {\bf Cloud clearing: -- }
the cloud-clearing scheme extends that developed by Saunders and 
Kriebel (1988) for use with AVHRR data over the North Atlantic.
Additional tests involve: 
consistency checks on forward and nadir brightness temperatures for 
individual channels; 
consistency checks on nadir-view brightness temperatures from different 
channels, and the use of the 1.6\mic\ channel in daytime.
Note that the requirements of the nadir/forward-view difference test have been 
relaxed from the ideal in order to avoid rejecting large quantities of 
data affected by aerosols from the 1991  Mount~Pinatubo eruption.
This can result in SSTs depressed by up to 1K from their true value
(and also in the admission of genuinely cloudy data). 
The error is greater in nadir-only  SSTs than in dual-view SSTs
by an amount which is correlated with the degree of atmospheric aerosol
contamination.
See Zavody {\etal} (1995) for a discussion of this and a description
of the cloud tests.

\item {\bf SST retrieval: --}
the SST retrieval scheme follows that demonstrated by McClain {\etal} 
(1985), where SST is given by a linear combination of brightness 
temperatures.
$$ {\rm SST} = a_0 + \sum^N_{i=1} a_i T_i$$
where $a_i$ is a constant linear regression coefficient and $T_i$ is the 
cloud-free scene brightness temperature observed by ATSR.
The coefficients are derived using global radiosonde sets and 
an atmospheric radiative transfer model; three different sets of 
coefficients are used depending on latitude range (polar, mid-latitude 
or tropical). 
Day-time SST retrieval uses  the 11\mic\ and 12\mic\ brightness temperatures;
night-time SST retrieval also uses the 3.7\mic\ brightness temperature 
when available (that is, for data acquired previous to the May~1992 failure).
Therefore a maximum of six brightness temperatures can contribute to an
SST derivation, {\sl i.e.\/} forward and nadir values from the 11, 12 
and 3.7\mic\  channels.
This calculation is performed independently for each pixel when
evaluating SSTs at 1{\thinspace}km$\times$1{\thinspace}km resolution,
or on averaged ten-arcminute brightness temperatures 
when evaluating half-degree ASSTs (see Section~\ref{subsec:sst_types}).
A complete description of the SST retrieval is given by
Z\'{a}vody \etal\ (1994, 1995).
\end{description}


All data contained on the CD-ROM set discussed here were produced using 
SADIST~v500.
However,  a radically-revised version of the processing
software, SADIST-2 is in the final stages of development.
In the new ASST product, SSTs averaged over a ten-arcminute cell 
will be stored; that is, nine values for each half-degree cell.
The implications of SADIST-2 for the ASST product and the new
ASST format made necessary by this change are discussed in 
Bailey (1995). 
Later versions of the ASST archive produced using SADIST-2, will
be made available on future CD-ROM sets.

\clearpage

\section{Global coverage and the ERS-1 orbit}
\label{app:orbit}

ERS-1 is in a near-polar, Sun-synchronous orbit, of inclination
98.5{\degree} and mean altitude 785{\thinspace}km.
The ATSR swath is approximately 256{\thinspace}km each side of the 
sub-satellite track;
this gives rise to near-global coverage in three days during the 
basic ERS-1 three-day repeat cycle, although some regions at tropical latitudes 
are never observed.
Complete global coverage is achieved during the alternative 35-day and 
168-day repeat cycles 
which have also been employed by ERS-1.

The result of a month's observation in January~1992 and January~1993
are shown in Figure~\ref{fig:repeat}; these months were associated 
with a 3-day and a 35-day repeat cycle respectively.
Note the characteristic gap pattern evident in the earlier map 
compared to the full coverage exhibited by the later one.
\begin{figure}
\caption{Global coverage during 3-day and 35-day repeat cycles}
\label{fig:repeat}
\vspace{10cm}
\end{figure}

\clearpage
A record of ERS-1 repeat cycles is given in Table~\ref{tab:cycles}.
Note that the different three-day repeat cycles were associated with
different global coverage, that is, the data voids were in  different places.

\begin{table}
\caption{\bf ERS-1 Repeat Cycles }
\label{tab:cycles}
\begin{center}
\begin{tabular}{|l|l|} \hline
{\bf Date range}                     & {\bf Repeat Cycle} \\ \hline \hline
31 July 1991 -- 10 December 1991     & 3-day (Commissioning Phase)\\
10--26 December 1991                 & ERS-1 orbit manoeuvres \\
26 December 1991 -- 30 March 1992    & 3-day (Ice Phase)\\
30 March 1992  -- 14 April 1992      & ERS-1 orbit manoeuvres \\
14 April 1992 -- 17 December 1993    & 35-day (Global Phase) \\ 
17--21 December 1993                 & ERS-1 orbit manoeuvres \\
21 December 1993 -- 10 April 1994    & 3-day (Second Ice Phase) \\
10 April 1994 -- 19 March 1995       & 168-day (Geodetic Phase) \\ 
19--21 March 1995                    & ERS-1 orbit manoeuvres \\
21 March 1995 --                     & 35-day (Global Phase)\\ \hline
\end{tabular}
\end{center}
\end{table}

\clearpage




{\footnotesize
\begin{thebibliography}{99}

\bibitem{sad:v500prod} P.~Bailey,  {\sl SADIST Products (Version 500)},
Space Science Department, Rutherford Appleton Laboratory, 1993. 

\bibitem{sad:sad2}  P.~Bailey, {\sl SADIST-2, v100 products},  
(ER-TN-RAL-AT-2164), 
Space Science Department, Rutherford Appleton Laboratory, 1995.

\bibitem{barton} I.J.~Barton \etal, ``Validation of the ATSR in 
Australian Waters'', 
{\sl J. Atmos. Oceanic Technol.}, {\bf 12}, 290-300.

\bibitem{bottomley} M.~Bottomley \etal, ``Global Ocean Surface 
Temperature Atlas, Joint Project, United Kingdom Meteorological Office and 
Massachusetts Institute of Technology'', 
{\sl Her Majesty's Stationary Office, London}, 20pp, 313 plates.

\bibitem{edwards} T.~Edwards \etal, ``The Along Track Scanning Radiometer
-- Measurement of Sea-Surface Temperature from ERS-1'',
{\sl J. British Interplan. Soc.}, {\bf 43}, 160--180,
1990.

\bibitem{jones} M.S.~Jones \etal, ``Reducing Cloud Contamination
in ATSR Averaged SST data'', 
{\sl J. Atmos. Oceanic Technol.}, accepted, 1995.

\bibitem{mcclain} E.P.~McClain \etal, ``Comparative performance of AVHRR 
based multichannel SSTs'', 
{\sl J. Geophys. Res.,} {\bf 90}, 11587--11601, 1985. 

\bibitem{mutlow} C.T.~Mutlow \etal, ``Sea surface temperature measurements
by the along-track scanning radiometer on the ERS-1 satellite: Early
results'', {\sl J. Geophys. Res.,} {\bf 99}, 575--588, 1994. 

\bibitem{Saunders} R.~W.~Saunders and K.~T.~Kriebel,
``An improved method for detecting clear sky and cloudy sky radiances
from AVHRR data'', 
{\sl Int. J. Remote Sens.,} {\bf 9}, 123--150, 1988.

\bibitem{zavodya} A.~Z\'{a}vody \etal, 
``The  ATSR data processing Scheme Developed for the  EODC'', 
{\sl Int. J. Remote Sens.,} {\bf 15}, 827--843, 1994.

\bibitem{zavodyb} A.~Z\'{a}vody \etal, 
``A radiative transfer model for SST retrieval for the ATSR'',
{\sl J. Geophys. Res.,} {\bf 100}, 937--952, 1995.

\bibitem{zavodyc} A.~Z\'{a}vody, private communication, 1995.

\end{thebibliography} }

Further information on ATSR can be obtained from:

Dr.\ Chris Mutlow,\\
ATSR-1 Principal Investigator,\\
Space Science Department,\\
Rutherford Appleton Laboratory,\\
Chilton, Didcot,\\
OX11 OQX,\\
United Kingdom.\\
~\\
Tel: 01235-446525\\
Fax: 01235-445848.

\bigskip

{\large\bf Acknowledgements:}
Thanks to Chris Mutlow, Paul Bailey, Peter Allan, Myles Allen, Peter Chiu, 
Brian Coan, Malcolm Currie, Sean Lawrence, Chunkey Lepine, 
David Llewellyn-Jones, Nigel Houghton, Chris Seelig, Phil Watts 
and Albin Z\'{a}vody for useful discussions on the data and software
presented here.
\end{document}

